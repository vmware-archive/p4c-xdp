\section{Experimental results}\label{sec:results}

\subsection{Testbed}
All of our performance results use a hardware testbed that consists of
two Intel Xeon E5 2440 servers, each with an Intel 10GbE X540-AT2 dual
port NIC, with the two ports of the Intel NIC on one server connected
to the two ports on the identical NIC on the other server.i
We installed p4c-xdp on one server, the {\em target server}, and
attached the XDP program to the port that receives the packets
The other server, the {\em source server}, generates packets
at the maximum 10~Gbps packet rate of 14.88~Mpps using the DPDK-based
TRex~\cite{trex} traffic generator.  The source server sends minimum
length 64-byte packets in various traffic patterns to one port of the
target server, and receives the forwarded packets on the other port.
At the target server, every packet received goes through the
pipeline specified in P4.
We measure the forwarding speed to evaluate the effectiveness of the
generated XDP program.

\subsection{Performance}
We measured the sample programs under the tests directory.

\begin{table}
\centering
\small
\begin{tabular}{lll}
  \underline{P4 program} & \underline{Description} & \underline{Performance} \\
  xdp1.p4 & parse L2/L3, drop if non-IP packet &  Mpps \\
  xdp2.p4 & parse L2/L3/L4, drop if non-IP packet &  Mpps \\
  xdp3.p4 & parse L2/L3, lookup &  Mpps \\
\end{tabular}
\caption{\footnotesize performance of XDP program generated by
  p4c-xdp compiler.}
\label{table:treebuild}
\end{table}
