\section{Introduction}\label{sec:introduction}

This section is adapted from~\cite{budiu-osr17}.

One of the most active areas in computer networking is Software
Defined Networking (SDN)~\cite{rfc7426}.  SDN separates the two core
functions of a network element (e.g., router): the control-plane and
the data-plane.  Traditionally both these functions were implemented
on the same device; SDN decouples them, and allows multiple
control-plane implementations for managing each data-plane.  A
standard SDN example is the Open Flow protocol~\cite{mckeown-ccr08},
which specifies the API between the control-plane and the data-plane.

Despite the additional flexibility brought by separating these
functions, SDN still assumes that the behavior of the network
data-plane is fixed.  This is a significant impediment to innovation;
for example, the deployment of the VXLAN protocol~\cite{rfc7348} took
4 years between the initial proposal and its commercial availability
in high-speed devices.

As a reaction to this state of affairs there is a new impetus to make
computer networks even more programmable by making the behavior of the
\emph{data-plane} expressible as software.  \cite{bosshart-ccr14} that
proposed the P4 language: Programming Protocol-independent Packet
Processors.  P4 gained rapid adoption.

The P4 consortium p4.org~\cite{p4org} was created to steward the
language evolution; p4.org currently includes more than 100
organizations in the areas of networking, cloud systems, network chip
design, and academic institutions.  The P4 specification is open and
public~\cite{p416-spec17}.  Reference implementations for compilers,
simulation and debugging tools are available with a permissive license
at the GitHub P4 repository~\cite{p4lang}.  While initially P4 was
designed for programming network switches, its scope has been
broadened to cover a large variety of packet-processing systems (e.g.,
network cards, FPGAs, etc.).
