\documentclass[10pt]{article}

\usepackage{caption}
\usepackage{float}
\usepackage{courier}
\usepackage{xspace}
\usepackage{listings}
\usepackage{graphicx}
\usepackage{mdframed}

\lstset{%basicstyle=\ttfamily
  language=C,
  basicstyle=\ttfamily\footnotesize,
  frame=lrbt,
  morekeywords={action,apply,bit,bool,%
const,control,default,else,%
enum,error,extern,exit,%
false,header,header_union,if,%
in,inout,int,match_kind,%
package,parser,out,return,%
select,state,struct,switch,%
table,transition,true,tuple%
typedef,varbit,verify,void,%
%
abstract,interface,class,virtual% used for IR
}
}
\newcommand{\PFOUR}{P4\xspace}
\newcommand{\code}[1]{\texttt{#1}}
\newcommand{\keyword}[1]{{\bf \texttt{#1}}}
\newcommand{\vonemodel}{\code{v1model}\xspace}

\title{Linux network programming with \PFOUR}
\author{William Tu\\
  VMware Inc.\\
  \texttt{tuc@vmware.com}
  \and
  Fabian Ruffy\\
  University of British Columbia\\
  \texttt{fruffy@cs.ubc.ca}
  \and
  Mihai Budiu\\
  VMware Research\\
  \texttt{mbudiu@vmware.com}
}
\date{}

\begin{document}
\maketitle

\begin{abstract}
  \PFOUR is a programming language for implementing network dataplanes.
\end{abstract}

\section{Introduction}\label{sec:introduction}

This section is adapted from~\cite{budiu-osr17}.

One of the most active areas in computer networking is Software
Defined Networking (SDN)~\cite{rfc7426}.  SDN separates the two core
functions of a network element (e.g., router): the control-plane and
the data-plane.  Traditionally both these functions were implemented
on the same device; SDN decouples them, and allows multiple
control-plane implementations for managing each data-plane.  A
standard SDN example is the Open Flow protocol~\cite{mckeown-ccr08},
which specifies the API between the control-plane and the data-plane.

Despite the additional flexibility brought by separating these
functions, SDN still assumes that the behavior of the network
data-plane is fixed.  This is a significant impediment to innovation;
for example, the deployment of the VXLAN protocol~\cite{rfc7348} took
4 years between the initial proposal and its commercial availability
in high-speed devices.

As a reaction to this state of affairs there is a new impetus to make
computer networks even more programmable by making the behavior of the
\emph{data-plane} expressible as software.  \cite{bosshart-ccr14} that
proposed the P4 language: Programming Protocol-independent Packet
Processors.  P4 gained rapid adoption.

The P4 consortium p4.org~\cite{p4org} was created to steward the
language evolution; p4.org currently includes more than 100
organizations in the areas of networking, cloud systems, network chip
design, and academic institutions.  The P4 specification is open and
public~\cite{p416-spec17}.  Reference implementations for compilers,
simulation and debugging tools are available with a permissive license
at the GitHub P4 repository~\cite{p4lang}.  While initially P4 was
designed for programming network switches, its scope has been
broadened to cover a large variety of packet-processing systems (e.g.,
network cards, FPGAs, etc.).

\section{Background}\label{sec:bacground}

\subsection{P4}

\subsection{eBPF for network processing}

\subsection{XDP: eXpress data path}

\subsection{Comparison of P4 and eBPF}

\section{Compiling P4 to eBPF}\label{sec:compilation}

\subsection{Packet filters with eBPF}

\cite{p4-ebpf-backend}

The following is the architectural model of an eBPF packet filter
expressed in P4.

\begin{lstlisting}
#include <core.p4>

extern CounterArray {
    CounterArray(bit<32> max_index, bool sparse);
    void increment(in bit<32> index);
}

extern array_table {
    array_table(bit<32> size);
}

extern hash_table {
    hash_table(bit<32> size);
}

parser parse<H>(packet_in packet, out H headers);
control filter<H>(inout H headers, out bool accept);

package ebpfFilter<H>(parse<H> prs,
                      filter<H> filt);
\end{lstlisting}

The following program shows a P4 program that counts the number of
IPv4 packets that are processed.

\begin{lstlisting}
#include <core.p4>
#include <ebpf_model.p4>

typedef bit<48> EthernetAddress;
typedef bit<32> IPv4Address;

header Ethernet_h {
    EthernetAddress dstAddr;
    EthernetAddress srcAddr;
    bit<16> etherType;
}

// IPv4 header without options
header IPv4_h {
    bit<4>       version;
    bit<4>       ihl;
    bit<8>       diffserv;
    bit<16>      totalLen;
    bit<16>      identification;
    bit<3>       flags;
    bit<13>      fragOffset;
    bit<8>       ttl;
    bit<8>       protocol;
    bit<16>      hdrChecksum;
    IPv4Address  srcAddr;
    IPv4Address  dstAddr;
}

struct Headers_t {
    Ethernet_h ethernet;
    IPv4_h     ipv4;
}

parser prs(packet_in p, out Headers_t headers) {
    state start {
        p.extract(headers.ethernet);
        transition select(headers.ethernet.etherType) {
            16w0x800 : ip;
            default : reject;
        }
    }

    state ip {
        p.extract(headers.ipv4);
        transition accept;
    }
}

control pipe(inout Headers_t headers, out bool pass) {
    CounterArray(32w10, true) counters;

    apply {
        if (headers.ipv4.isValid()) {
            counters.increment((bit<32>)headers.ipv4.dstAddr);
            pass = true;
        } else
            pass = false;
    }
}

ebpfFilter(prs(), pipe()) main;
\end{lstlisting}

\subsection{Packet forwarding with XDP}

The following is the architectural model of an XDP-based packet switch
expressed in P4.

\cite{p4-xdp-backend}

\begin{lstlisting}
#include <ebpf_model.p4>
enum xdp_action {
  XDP_ABORTED,
  XDP_DROP,
  XDP_PASS,
  XDP_TX
}
struct xdp_input { bit<32> input_port }

struct xdp_output {
  xdp_action output_action;
  bit<32> output_port;
}
parser xdp_parse<H>(packet_in packet, out H headers);
control xdp_switch<H>(inout H hdrs, in xdp_input i, out xdp_output o);
control xdp_deparse<H>(in H headers, packet_out packet);

package xdp<H>(xdp_parse<H> p, xdp_switch<H> s, xdp_deparse<H> d);
\end{lstlisting}

\section{Testing eBPF programs}\label{sec:testing}
To verify the correctness of a P4-XDP program, the compiler integrates a full 
end-to-end testing framework. The framework consists of an userspace eBPF 
runtime as well as a kernel testing pipeline, which verifies eBPF/XDP programs 
in a virtual, isolated environment.

\subsection{Why Test in User-Space?}
\begin{itemize}
	\item Test the compiler correctness
	\item Isolate functionality in user-space
	\item C output must be functionally equivalent to P4
	\item Fewer dependencies. Does not require specific versions of Linux 
	kernel llvm Kernel hook or on eBPF verifier
	\item Debugging simplicity (e.g., GDB, printf, valgrind)
\end{itemize}

\subsection{The Simple Test Framework}
\begin{figure}
	\centering
	\includegraphics[width=0.7\linewidth]{stf}
	\caption{}
	\label{fig:stf}
\end{figure}
The simple test framework is a data plane verification language, which is used 
by the p4c-ebpf compiler. While its original purpose is to assess switching and 
forwarding behavior, it can also test the functionality of eBPF programs in 
isolation.
In general, an .stf file describes the input as well the expected output 
packets per switch, or in this case virtual bridge, port. In addition, it may 
also define the initial state of the dataplane tables and may contain a list of 
control plane commands which are to be run in sequence....

Compiler comes with a Python parser for STF

\subsection{The Test Runtime}
Describe the userspace runtime.
User-space wrappers around eBPF tables and APIs
\begin{figure}
	\centering
	\includegraphics[width=0.7\linewidth]{user_test}
	\caption{}
	\label{fig:user_test}
\end{figure}

\subsection{Testing P4 Programs End-to-End}
Talk about how to use the test framework to verify your eBPF code or p4 file 
end-to-end. Highlight, how the framework can be used to test eBPF and XDP 
programs in general, independent from the P4 pipeline.
\begin{figure}
	\centering
	\includegraphics[width=0.7\linewidth]{kernel_test}
	\caption{}
	\label{fig:kernel_test}
\end{figure}
\section{Experimental results}\label{sec:results}
\subsection{Testbed}
All of our performance results use a hardware testbed that consists of
two Intel Xeon E5 2440 v2 1.9GHz servers, each with an Intel 10GbE X540-AT2 dual
port NIC, with the two ports of the Intel NIC on one server connected
to the two ports on the identical NIC on the other server.i
We installed p4c-xdp on one server, the {\em target server}, and
attached the XDP program to the port that receives the packets
The other server, the {\em source server}, generates packets
at the maximum 10~Gbps packet rate of 14.88~Mpps using the DPDK-based
TRex~\cite{trex} traffic generator.  The source server sends minimum
length 64-byte packets in {\em single} UDP flow to one port of the
target server, and receives the forwarded packets on the same port.
At the target server, every packet received goes through the
pipeline specified in P4.

We use sample p4 programs under the tests directory and the following
metrics to understand the performance impact of the P4-generated XDP
program:
\begin{itemize}
\item Packet Processing Rate (Mpps): Once XDP program finishes processing
the packet, it returns one of the actions mentioned in section~\ref{background}.
We made a small modification to all p4 program to always return XDP\_DROP,
so that we can count the number of packets being drop per second as a
indication of how fast the XDP can process.
\item CPU Utilization: Every packets processed by XDP program is run
under the per-core software IRQ daemon, named ksoftirqd/<core id>.
We measure the CPU utilization of the ksoftirqd on the core.
\item Number of BPF instructions verified: For each program, we list
the its max complexity; the total number of BPF instructions the
verifier has to go through, as an indication of how complicated the
program is.
\end{itemize}

The target server is running Linux kernel 4.19-rc5 and for all our
tests, the BPF JIT (Just-In-Time) compiler is enabled and JIT harden
is disabled. All programs are compiled with clang 3.8 with llvm 5.0.
For For each test program, we use the following
command from iproute2 tool to load it into kernel:
%\begin{verbatim}
\texttt{ip link set dev eth0 xdp obj xdp1.o verb}.
%\end{verbatim}

The Intel 10GbE X540 NIC is running ixgbe driver with 16 RX queues
set-up. Since the source server is sending single UDP flow, packets
always arrive at a single ring ID.  As a result, we collect the number
of packets being dropped at this ring.

\subsection{Results}
To compare the performance, we first started by manually writing two
XDP programs. The first one, SimpleDrop, does nothing but drop all packets by
returning XDP\_DROP. The second program also does nothing but returns
XDP\_TX, which forwards the packet to the receiving port.  Both programs
consists of only two BPF instructions.

{\small
\begin{verbatim}
    /* SimpleDrop */
    0: (b7) r0 = 1 // XDP_DROP
    1: (95) exit

    /* SimpleTX */
    0: (b7) r0 = 3 // XDP_TX
    1: (95) exit
\end{verbatim}
}
Then We attached the following P4 programs to the device receiving the
rate of 14.88~Mpps to evaluate the overhead introduce by the P4C-XDP
compiler.
\begin{itemize}
\item xdp1.p4: Parse Ethernet/IPv4 header, deparse it, and drop.
\item xdp3.p4: Parse Ethernet/IPv4 header, lookup an mac address
table, deparse it, and drop.
\item xdp6.p4: Parse Ethernet/IPv4 header, lookup and get a new TTL value
from eBPF map, set to IPv4 header, deparse it, and drop.
\item xdp7.p4: Parse Ethernet/IPv4/UDP header, write a pre-defined source port
and source IP, recalculate checksum, deparse, and drop.
\item xdp11.p4: Parse Ethernet/IPv4 header, swap src/dst mac address,
deparse it, and send back to the same port (XDP\_TX).
\item xdp15.p4: Parse Ethernet header, insert a customized 8-byte header,
deparse it, and send back to the same port (XDP\_TX).
\end{itemize}

\begin{table}
\centering
\small
\begin{tabular}{llll}
  \underline{P4 program} & \underline{CPU Util.} & \underline{Mpps} & \underline{Insns./Stack}\\
  SimpleDROP & 75\% & 14.4 & 2/0 \\
  SimpleTX & 100\% & 7.2 & 2/0 \\
  xdp1.p4 &  100\% &  8.1 & 277/256 \\
  xdp3.p4 &  100\% &  7.1 & 326/256 \\
  xdp6.p4 &  100\% &  2.5 & 335/272 \\
  xdp7.p4 &  100\% &  5.7 & 5821/336 \\
  xdp11.p4 &  100\% &  4.7  & 335/216 \\
  xdp15.p4 &  100\% &  5.5 & 96/56\\
\end{tabular}
\caption{\footnotesize Performance of XDP program generated by
  p4c-xdp compiler.}
\label{tab:perf}
\end{table}

As shown in Table~\ref{tab:perf}, the xdp1.p4 shows the baseline overhead
introduced by adding the parser and deparser, dropping the rate from 14.4~Mpps to
8.1~Mpps. The xdp3.p4 drops another million packet per second due to
calling the eBPF map lookup function to do a lookup, the lookup is designed
to always return NULL so no value from the map is accessed.
The xdp6.p4 shows significant overhead because it is designed to lookup
a table, find a new TTL value, and write to the IPv4 header. 
Surprisingly, the xdp7.p4 does extra parsing to the UDP header and
checksum recalculation, but the overhead is moderate due to not accessing
the table.

Finally, the xdp11.p4 and xdp15.p4 shows the transmit (XDP\_TX) performance.
Compared with xdp11 and xdp15, the xdp15.p4 involves the extra bpf\_adjust\_head helper
function to reset the pointer for extra bytes.  
Interestingly, it does not incur much over head because there is
already a reserved space in front of every XDP packet frame.

\subsection{Microbenchmark}
To further understand the performance overhead of programs generated by p4c-xdp,
we manually {\emph comments out} the entire deparser portion of the C code, so that
we can identify which stage of our XDP program (parser, lookup, and deparser) incurs
the performance overhead.

\begin{table}
\centering
\small
\begin{tabular}{llll}
  \underline{P4 program} & \underline{CPU Util.} & \underline{Mpps} & \underline{Insns./Stack}\\
  xdp1.p4 &  77\% &  14.8 & 26/0 \\
  xdp3.p4 &  100\% &  13 & 100/16 \\
  xdp6.p4 &  100\% &  12 & 98/40 \\
\end{tabular}
\caption{\footnotesize Performance of XDP program without deparser.}
\label{tab:perf2}
\end{table}

As shown in Table~\ref{tab:perf2}, the performance increases significantly.
By investigating our p4c-xdp compiler implementation and the generated C
code, we figured out that the deparser is unconditionally writing back
the entire packet content even when the P4 program does not modify any.
In addition, the deparser incurs lots of byte-order translation, e.g.,
htonl, ntohl. This could be avoided by always using network byte-order
in P4 and XDP. We leave this for future optimization.

\section{Challenges}\label{sec:conclusions}
In general, our development experience is mirrored by the lessons described 
in~\cite{minao-hspr18} and ~\cite{bertin-netdev17}.

\paragraph{No multi-/broadcast support}
While XDP is able to redirect single frames it does not have the ability to 
clone and redirect packets similar to \texttt{bpf\_clone\_redirect}. This makes 
development of more sophisticated P4 forwarding programs problematic.
\paragraph{The stack size is to small}
More complex XDP programs are rejected by the verifier despite their safeness. 
This is a particular challenge when attempting to implement network function 
chaining or more advanced pipelined packet processing in a single XDP program.

\paragraph{XDP generic and TCP}
Our testing framework uses virtual Linux interfaces and XDP \texttt{generic} 
to verify XDP programs. 
Unfortunately, we are unable to test TCP streams as the protocol is not 
supported by this driver.
Any program loaded by XDP \texttt{generic} operates after the creation of 
the \texttt{skb} and requires the original packet data. Since TCP clones every 
packet and passes the unmodifiable \texttt{skb} clone,  XDP \texttt{generic} is 
bypassed and never receives the datagram.
\paragraph{Missing userspace library}
Creating of compilation of eBPF programs in userspace requires substantial 
effort. Many function calls and variables available in sample programs are not 
available as C library and have to copied from kernel code or assembled from 
various online sources.

\paragraph{Persistent eBPF maps in namespaces}
[need to verify] When using eBPF programs in namespaces, maps exported via tc 
do not persist across \texttt{ip netns exec} calls. The consequence is that any 
program has to be run in a single shell command, otherwise the eBPF map becomes 
unusable despite the continued existence of the namespace.

\end{document}
